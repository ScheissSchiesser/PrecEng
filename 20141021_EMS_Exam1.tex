\documentclass{article}
% Change "article" to "report" to get rid of page number on title page
\usepackage{amsmath,amsfonts,amsthm,amssymb}
\usepackage{setspace}
\usepackage{Tabbing}
\usepackage{fancyhdr}
\usepackage{lastpage}
\usepackage{extramarks}
\usepackage{chngpage}
\usepackage{soul,color}
\usepackage{graphicx,float,wrapfig}
\usepackage{IEEEtrantools}
\usepackage[]{mcode}
\usepackage{url}
\usepackage{siunitx}

% In case you need to adjust margins:
\topmargin=-0.45in      %
\evensidemargin=0in     %
\oddsidemargin=0in      %
\textwidth=6.5in        %
\textheight=9.0in       %
\headsep=0.25in         %

% Homework Specific Information
\newcommand{\hmwkTitle}{Exam 1}
\newcommand{\hmwkDueDate}{10/21/2014}
\newcommand{\hmwkClass}{Precision Engineering}
\newcommand{\hmwkClassTime}{}
\newcommand{\hmwkClassInstructor}{Prof. Ellis}
\newcommand{\hmwkAuthorName}{Eric Schiesser}

% Setup the header and footer
\pagestyle{fancy}                                                       %
\lhead{\hmwkAuthorName}                                                 %
\chead{\hmwkClass\ (\hmwkClassInstructor\ \hmwkClassTime): \hmwkTitle}  %
\rhead{\firstxmark}                                                     %
\lfoot{\lastxmark}                                                      %
\cfoot{}                                                                %
\rfoot{Page\ \thepage\ of\ \pageref{LastPage}}                          %
\renewcommand\headrulewidth{0.4pt}                                      %
\renewcommand\footrulewidth{0.4pt}                                      %

% This is used to trace down (pin point) problems
% in latexing a document:
%\tracingall

%%%%%%%%%%%%%%%%%%%%%%%%%%%%%%%%%%%%%%%%%%%%%%%%%%%%%%%%%%%%%
% Some tools
\newcommand{\enterProblemHeader}[1]{\nobreak\extramarks{#1}{#1 continued on next page\ldots}\nobreak%
                                    \nobreak\extramarks{#1 (continued)}{#1 continued on next page\ldots}\nobreak}%
\newcommand{\exitProblemHeader}[1]{\nobreak\extramarks{#1 (continued)}{#1 continued on next page\ldots}\nobreak%
                                   \nobreak\extramarks{#1}{}\nobreak}%

\newlength{\labelLength}
\newcommand{\labelAnswer}[2]
  {\settowidth{\labelLength}{#1}%
   \addtolength{\labelLength}{0.25in}%
   \changetext{}{-\labelLength}{}{}{}%
   \noindent\fbox{\begin{minipage}[c]{\columnwidth}#2\end{minipage}}%
   \marginpar{\fbox{#1}}%

   % We put the blank space above in order to make sure this
   % \marginpar gets correctly placed.
   \changetext{}{+\labelLength}{}{}{}}%

\setcounter{secnumdepth}{0}
\newcommand{\homeworkProblemName}{}%
\newcounter{homeworkProblemCounter}%
\newenvironment{homeworkProblem}[1][Problem \arabic{homeworkProblemCounter}]%
  {\stepcounter{homeworkProblemCounter}%
   \renewcommand{\homeworkProblemName}{#1}%
   \section{\homeworkProblemName}%
   \enterProblemHeader{\homeworkProblemName}}%
  {\exitProblemHeader{\homeworkProblemName}}%

\newcommand{\problemAnswer}[1]
  {\noindent\fbox{\begin{minipage}[c]{\columnwidth}#1\end{minipage}}}%

\newcommand{\problemLAnswer}[1]
  {\labelAnswer{\homeworkProblemName}{#1}}

\newcommand{\homeworkSectionName}{}%
\newlength{\homeworkSectionLabelLength}{}%
\newenvironment{homeworkSection}[1]%
  {% We put this space here to make sure we're not connected to the above.
   % Otherwise the changetext can do funny things to the other margin

   \renewcommand{\homeworkSectionName}{#1}%
   \settowidth{\homeworkSectionLabelLength}{\homeworkSectionName}%
   \addtolength{\homeworkSectionLabelLength}{0.25in}%
   \changetext{}{-\homeworkSectionLabelLength}{}{}{}%
   \subsection{\homeworkSectionName}%
   \enterProblemHeader{\homeworkProblemName\ [\homeworkSectionName]}}%
  {\enterProblemHeader{\homeworkProblemName}%

   % We put the blank space above in order to make sure this margin
   % change doesn't happen too soon (otherwise \sectionAnswer's can
   % get ugly about their \marginpar placement.
   \changetext{}{+\homeworkSectionLabelLength}{}{}{}}%

\newcommand{\sectionAnswer}[1]
  {% We put this space here to make sure we're disconnected from the previous
   % passage

   \noindent\fbox{\begin{minipage}[c]{\columnwidth}#1\end{minipage}}%
   \enterProblemHeader{\homeworkProblemName}\exitProblemHeader{\homeworkProblemName}%
   \marginpar{\fbox{\homeworkSectionName}}%

   % We put the blank space above in order to make sure this
   % \marginpar gets correctly placed.
   }%

%%%%%%%%%%%%%%%%%%%%%%%%%%%%%%%%%%%%%%%%%%%%%%%%%%%%%%%%%%%%%


%%%%%%%%%%%%%%%%%%%%%%%%%%%%%%%%%%%%%%%%%%%%%%%%%%%%%%%%%%%%%
% Make title
\title{\vspace{2in}\textmd{\textbf{\hmwkClass:\ \hmwkTitle}}\\\normalsize\vspace{0.1in}\small{Due\ on\ \hmwkDueDate}\\\vspace{0.1in}\large{\textit{\hmwkClassInstructor\ \hmwkClassTime}}\vspace{3in}}
\date{}
\author{\textbf{\hmwkAuthorName}}
%%%%%%%%%%%%%%%%%%%%%%%%%%%%%%%%%%%%%%%%%%%%%%%%%%%%%%%%%%%%%

\begin{document}
\begin{spacing}{1.1}
\maketitle
\newpage
% Uncomment the \tableofcontents and \newpage lines to get a Contents page
% Uncomment the \setcounter line as well if you do NOT want subsections
%       listed in Contents
%\setcounter{tocdepth}{1}
%\tableofcontents
%\newpage

% When problems are long, it may be desirable to put a \newpage or a
% \clearpage before each homeworkProblem environment

\clearpage
\begin{homeworkProblem}


\begin{homeworkSection}{(a)}

A 3 V-groove kinematic mount where the lines forming each V meet in the center allows thermal expansion from the center, since each V allows one axis of translational motion. In a Cup-V-Flat mount, the cup allows no translational motion, so the expansion occurs about that point, causing the expansion to be asymmetric about the center. This could be a problem when mounting optics because it will cause boresight error. Also, depending on the orientation of the V in the Cup-V-Flat, the thermal expansion could cause a rotation about the Cup, since the sphere in the V must follow the single translation axis it allows. Thus, use the 3-Vee groove when concerned with thermal expansion. The Vees need to be precisely machined so that they will allow expansion from the center, so the Cup-V-Flat is the better/cheaper option when you don’t need to worry about thermal expansion about the center of the mount (for instance, you are mounting a plane mirror that is underfilled).


\end{homeworkSection}

\begin{homeworkSection}{(b)}
\begin{figure}
  \caption{Pressure vs. Angle.}
    \includegraphics[width=0.5\textwidth]{"C:/Users/Eric/Documents/PhD/Precision Engineering/Exam 1/pressurevsangle"}
\end{figure}

\begin{figure}
  \caption{Area vs. Angle.}
    \includegraphics[width=0.5\textwidth]{"C:/Users/Eric/Documents/PhD/Precision Engineering/Exam 1/areavsangle"}
\end{figure}

As we can see from the above plots, the highest pressure is at a steep angle, due to the direct relationship between the normal force $F_n$ and the contact pressure. Since $F_n$ increases with V-groove angle ($F_n=F \sin⁡(\theta)$), the contact pressure increases.

\end{homeworkSection}

\begin{homeworkSection}{(c)}

I assume that the vertical displacement of the sphere is the projection of the $\delta$ described in the notes: 

\begin{equation}
\delta = a^2/(2R^\star)
\end{equation}

% \lstinputlisting{"C:/Users/Eric/Google Drive/MATLAB/plotStress.m"}

\end{homeworkSection}

\end{homeworkProblem}

\clearpage
\begin{homeworkProblem}

\begin{homeworkSection}{(a)}
BK-7 has a CTE of $\alpha_{BK7} =  \SI{7.1e-6}{\per\kelvin}$ and aluminum has a CTE of $\alpha_{Al} = \SI{23.6e-6}{\per\kelvin}$. The adhesive has $\alpha_{RTV} =  \SI{200e-6}{\per\kelvin}$. A $\SI{100}{\milli\meter}$ lens will expand by $\SI{85.2}{\micro\meter}$ in a temperature range from $\SI{-40}{\celsius}$ to $\SI{-80}{\celsius}$ (under linear approximation). The glue will not stress the optic if we allow enough space between the lens diameter $d_g$ and the mount diameter $d_m$ according to:
\begin{equation}
h_r = \frac{d_g (\alpha_{Al} - \alpha_{BK7})}{2 (\alpha_{RTV} - \alpha_{Al})} = \SI{4.677}{\milli\meter}
\end{equation}
where $h_r$ is the difference between the lens semi-diameter and the mount semi-diameter.
\end{homeworkSection}

\begin{homeworkSection}{(b)}
If there is some cocentricity error between the optical axis and the physical axis, we can drill threaded holes and use nylon screws to push around the lens until the optical axis is centered correctly, and then inject the adhesive through injection holes to hold it in place. The screws could then be removed if necessary.

We could also use tangential or spherical seats on the mount to allow the optic to self-center, although these seats would have to be precisely machined and there would need to be enough room between the edges of the lens and the mount to allow for the lens to move to its optically centered position.

This cocentricity error would add time and cost to the assembly process because it requires a custom alignment of each optical element, which is time consuming and expensive because it requires a human.
\end{homeworkSection}

\begin{homeworkSection}{(c)}
Assuming we mount the mirror on 3 points at $0.645 r$ for minimum RMS self-weight deflection, we can use the following equation (from Katie Schwertz's thesis):
\begin{equation}
\delta_{RMS}=0.316 \frac{\rho g}{E} \frac{r^4}{t^2} (1-\nu^2)
\end{equation}

Solving this for the thickness $t$, we get:

\begin{equation}
t=0.5621 \sqrt{\frac{\rho g}{E} \frac{r^4}{\delta_{RMS}} (1-\nu^2)}
\end{equation}

We have $E= \SI{90.3}{\giga\pascal}$, $\rho = \SI{2.53}{\gram\per\cubic\centi\meter}$, and $\nu = 0.24$ for Zerodur. $\delta_{RMS} = \lambda/20 = \SI{633}{\nano\meter}/20 = \SI{31.7}{\nano\meter} $. These values give:
$$
t = \SI{36.2}{\milli\meter}
$$
which gives an estimated mass of:
$$
m = \pi r^2 t \rho = \SI{6.48}{\kilo\gram}
$$

To allow for thermal expansion of the base plate without distorting the mirror, we can mount the mirror onto the baseplate using a 3 V-groove setup, which will allow for thermal expansion about the center of the mount.

Source for Zerodur physical properties:
\url{http://www.us.schott.com/advanced_optics/english/syn/advanced_optics/products/optical-materials/zerodur-extremely-low-expansion-glass-ceramic/zerodur/index.html}
\end{homeworkSection}
\end{homeworkProblem}

\begin{homeworkProblem}
\begin{homeworkSection}{(a)}
Quadrature current takes advantage of the identity:
\begin{equation} \label{eq:quad}
\sin[2\pi ft + \phi(t)]\ =\ \underbrace{
\sin(2\pi ft)\cdot \cos[\phi(t)]}_{\text {in-phase}
}\ +\ \underbrace{
\overbrace{
\sin\left(2\pi ft + \tfrac{\pi}{2} \right)}^{\cos(2\pi ft)
}\cdot \sin[\phi(t)]
}_{\text {quadrature}}.
\end{equation}
% An applied sinusoidal voltage drives a sinusoidal current, with some phase difference $\phi$ between them. ~\ref{eq:quad} describes % the current signal. When the phase is such that the voltage and current are
to allow you to apply a sinusoidal current signal by summing two sinusoidal signals that are \ang{90} out of phase. The phase term $\phi$ allows the control of the velocity (forward, backward, and speed).
This is important because it allows stable control of linear motor speed without having to drive it with a constant DC signal, which is more efficient.

Equation ~\ref{eq:quad} from \url{http://en.wikipedia.org/wiki/In-phase_and_quadrature_components} (sorry, it already had the \LaTeX \space markup).
\end{homeworkSection}

\begin{homeworkSection}{(b)}
There is less current per channel with the 3-phase motor to achieve the same power as a single coil. You can also get a more consistent current profile with 3 phases vs. single phase.
\end{homeworkSection}

\begin{homeworkSection}{(c)}

\end{homeworkSection}

\begin{homeworkSection}{(d)}

\end{homeworkSection}
\end{homeworkProblem}

\begin{homeworkProblem}
\begin{homeworkSection}{(a)}
The efficiency is given by:
\begin{equation} \label{eq:efficiency}
e=\frac{\cos{(\alpha)} ( \pi \beta \cos{(\alpha)} - \mu}{\pi \beta \cos{(\alpha)} (\cos{(\alpha)} + \pi \beta \mu)}
\end{equation}
where $\beta=d/l=4$, $\alpha = \ang{15}$, $l=\SI{5}{\milli\meter}$, $\mu = 0.05$ and $d=\SI{20}{\milli\meter}$.
These values give an efficiency of $e=0.603$, or about $60\%$.
The resulting output force is given by:
\begin{equation}
F_z = \frac{2\pi e \mu_z}{l}
\end{equation}
Solving for torque, $\mu_z$:
\begin{equation}
\mu_z= \frac{l F_z}{2\pi e}
\end{equation}
To apply a force of $F_z = \SI{500}{\newton}$, we require a torque of $\SI{0.659}{\newton\meter}$.

\end{homeworkSection}

\begin{homeworkSection}{(b)}
The diameter required is given by solving equation ~\ref{eq:efficiency} for $\beta$ and subsequently $d$.
\begin{equation}
\beta = \frac{1}{2 \pi e \mu} \left[ (1-e) \cos{\alpha} \pm \sqrt{(1-e)^2 \cos^2{\alpha}-4 e \mu^2} \right]
\end{equation}
The resulting equation is the solution to a quadratic equation in $\beta$, and therefore we have two possible values which give an efficiency of $90\%$, $d = \SI{1.39}{ \milli\meter}$ and $d = \SI{2.03}{ \milli\meter}$.
\end{homeworkSection}

\begin{homeworkSection}{(c)}
Assuming no lubrication (not likely) and hard steel, the coefficient of friction is $\mu_{SS} = 0.78$, which gives an efficiency of $e_{SS} = 0.0839$, or about $8.4\%$.

With steel on oil-impregnated brass, the coefficient of friction is  $\mu_{SB} = 0.19 $ which gives $e_{SS} = 0.283$, or about $28\%$.

Coefficients of friction from \url{http://www.engineershandbook.com/Tables/frictioncoefficients.htm}
\end{homeworkSection}

\end{homeworkProblem}

\begin{homeworkProblem}
\begin{homeworkSection}{(a)}
\end{homeworkSection}
\end{homeworkProblem}

\end{spacing}
\end{document}

%%%%%%%%%%%%%%%%%%%%%%%%%%%%%%%%%%%%%%%%%%%%%%%%%%%%%%%%%%%%%

%----------------------------------------------------------------------%
% The following is copyright and licensing information for
% redistribution of this LaTeX source code; it also includes a liability
% statement. If this source code is not being redistributed to others,
% it may be omitted. It has no effect on the function of the above code.
%----------------------------------------------------------------------%
% Copyright (c) 2007, 2008, 2009, 2010, 2011 by Theodore P. Pavlic
%
% Unless otherwise expressly stated, this work is licensed under the
% Creative Commons Attribution-Noncommercial 3.0 United States License. To
% view a copy of this license, visit
% http://creativecommons.org/licenses/by-nc/3.0/us/ or send a letter to
% Creative Commons, 171 Second Street, Suite 300, San Francisco,
% California, 94105, USA.
%
% THE SOFTWARE IS PROVIDED "AS IS", WITHOUT WARRANTY OF ANY KIND, EXPRESS
% OR IMPLIED, INCLUDING BUT NOT LIMITED TO THE WARRANTIES OF
% MERCHANTABILITY, FITNESS FOR A PARTICULAR PURPOSE AND NONINFRINGEMENT.
% IN NO EVENT SHALL THE AUTHORS OR COPYRIGHT HOLDERS BE LIABLE FOR ANY
% CLAIM, DAMAGES OR OTHER LIABILITY, WHETHER IN AN ACTION OF CONTRACT,
% TORT OR OTHERWISE, ARISING FROM, OUT OF OR IN CONNECTION WITH THE
% SOFTWARE OR THE USE OR OTHER DEALINGS IN THE SOFTWARE.
%----------------------------------------------------------------------%